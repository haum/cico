\documentclass[a4paper,10pt]{scrartcl}
%\documentclass[a4paper,10pt]{scrartcl}

\usepackage[xetex]{graphicx}
\usepackage{fontspec}

\title{CICO}
\author{HAUM}
\date{26/02/2016}

\begin{document}
\maketitle

\hfill
\raisebox{-.5\height}{\includegraphics[width=6cm]{logo}}

\begin{minipage}{10cm}
Organisme/Structure : HAUM\\
Adresse : 19 boulevard Marie \& Alexandre Oyon\\
Prénom, Nom contact : Romuald Conty\\
Titre : Président\\
Tel portable : 06.60.09.66.20\\
Mail : romuald@haum.org\\
\end{minipage}


\section{Présentation de vos activités et en particulier par rapport à l’innovation}

Le HAUM est une association sarthoise qui structure un Hackerspace AU Mans, littéralement un lieu pour bidouilleurs.

En tant que tel, l'association anime un lieu de vie, un endroit où des passionnés, des curieux, des experts, etc. se retrouvent, collaborent et échangent. C'est également un lieu de découverte, d'apprentissage, d'expérimentation, de discussion et de partage.

En favorisant ainsi les rencontres de personnes, quelque soit leur bagage technique ou leurs objectifs, et en soutenant la transmission de savoirs et de savoir-faire, l'association a su attirer des profils variés dont les échanges ont mené à des projets innovants.

Par exemple en 2014, le HAUM réalise un jeu lumineux (Pong1D) pour l'évènement culturel "les siestes Tériaki" à l'abbaye de l'Épau. Le projet séduit le public par son caractère atypique et la qualité de la réalisation. L'association est désormais très sollicitée pour présenter ce projet, ce qui a permis par exemple de nouer des liens avec le fablab de Nantes (porté par l'association Ping) qui a demandé une participation du HAUM à leur "FestivalD", ainsi qu'avec d'autres hackerspaces et fablabs.

Aussi, la diversité des sensibilités et compétences présentes au sein de l'association, ainsi que les libertés totales accordées aux acteurs de la création, ont permis de concrétiser des projets qui n'auraient pas pu exister dans des structures plus codifiées. Des instruments de musique géants dont les visiteurs, en se déplaçant dans l’installation, constituent à la fois les musiciens et des éléments de l'instrument lui-même auraient-ils pu émerger sans une structure telle que elle proposée par le HAUM ?

Dans son action, l'association s'investit également régulièrement avec d'autres structures partageant l'envie de "faire" comme Teriaki pour des animations au cours des siestes 2014 puis du festival 2015, Ping pour le FestivalD à Nantes, la Ruche Numérique avec des actions de formations techniques, de présentation d'outils, de conférences, etc., l'ENSIM avec l'élaboration et l'animation de sujets pour les 24h du code, les conseils de quartiers avec une coorganisation de Repair Cafés, et bien d'autres.

En outre, le hackerspace ne se limite pas à des activités technologiques et explore des domaines contrastés venant enrichir le champ d'innovation. Indiquons par exemple le recyclage de cartons usagés pour la réalisation de mobilier, l'exploration de topographies mathématiques ayant donné lieu à la création de casses-têtes ingénieux, diverses expérimentations dans les domaines du design et de la photographie, etc.

Aujourd'hui le HAUM dispose d'une expérience réelle pour aider à capter et cristalliser des idées innovantes dont les réalisations effectives sont régulièrement plébiscités. La présence d'un lieu facilement accessible, disposant de matériel partagé, et favorisant les échanges entre publics hétéroclites (entrepreneurs, artistes, techniciens, passionnés...) semble important pour stimuler la créativité et l'innovation sur Le Mans. 

\section{Vos idées et commentaire sur l’utilisation des deux espaces repérés sur Novaxud}
\emph{(Bâtiment ex « chaufferie » de 380 m² et surface bureaux disponibles de 1 000 m²)}

Le HAUM est très favorable à l'utilisation de l'ex chaufferie.

Il est important de ne pas figer son aménagement et au contraire livrer un lieu brut, modulable, qui pourra être reconfiguré au gré des besoins dictés par les innovations. Tantôt lieu de fabrication, tantôt lieu d'exposition, il faut que le lieu puisse vivre et s'adapter aux utilisateurs de l'espace, pour des projets ponctuels ou de plus longue durée. Les membres HAUM sont prêts à aider au démantèlement et souhaitent participer à son aménagement avec l'ensemble des volontaires, afin de s'assurer de créer un lieu de vie à l'image de ses occupants.

D'un point de vue pratique, il serait utile de prévoir une grande entrée afin d'y faire entrer des machines volumineuses et souvent nécessaires dans de tels lieux. En effets, certains fablabs ont déjà été contraints d'effectuer de la maçonnerie pour installer des machines. Le prévoir dès le début du projet peut éviter les déconvenues.

En ce qui concerne l'espace de 1000m², il a l'avantage d'être déjà disponible pour débuter le projet. Rapidement, il serait cependant préférable de le remplacer par les deux bâtiments à coté de la chaufferie, plus propices au développement du projet, et offrant un cachet, une atmosphère à la Cité. 

\section{Résumé l'activité du HAUM}

Le HAUM est une association sarthoise créée il y a 3 ans qui axe son action autour de la transmission de savoirs et de savoir-faire. Cette structure propose à chacun, quelque soit son bagage technique, de venir rencontrer d'autres passionnés, par exemple lors de séances ouvertes au public

Depuis 2 ans, l'activité de l'association a pris de l'ampleur. Elle s'adresse à une audience assez large au travers de projets accessibles aux néophytes : pong 1D (jeu), PianoStairs et dHAUM (instruments de musique géants et collaboratifs), meubles en carton, etc.

Dans son action, le HAUM s'associe régulièrement avec des structures partageant l'envie de "faire" : Teriaki (pour les siestes 2014 puis le festival 2015), Ping (pour FestivalD, à Nantes), la Ruche Numérique (hébergement, formations, conférences, etc.), ENSIM (24h du code), Conseils de Quartiers (Repair Cafés) et continuera : Gamer Assembly 2016 (Poitiers), etc.

Forte de ces interactions, pleine de projets variés et originaux, l'association compte aujourd'hui une trentaine de membres et poursuit toujours ses deux objectifs principaux. Le premier est de fournir à tous un environnement accueillant pour apprendre, découvrir et partager. Cet objectif s'articule autour de projets communs qui avancent avec la communauté ou de projets personnels ayant besoin d'aide extérieure.  

Le second objectif est la diffusion des connaissances. À ce titre, le HAUM organise des cycles de conférences courtes pour que chacun puisse présenter un sujet qui lui tient à cœur et que tous puissent découvrir des sujets en dehors de leur intérêt direct. De même, pour les sujets demandant plus de temps d'appréhension, l'association a mis en place des sessions de formation et à l'occasion des ateliers.

Tous ces points rejoignent un idéal d'auto-formation, d'apprentissage des techniques au travers de l'expérimentation et au travers de la rencontre avec d'autres. Cet aspect est essentiel dans un projet global de \emph{Cité de l'Innovation Collaborative} car il apporte le renouveau permanent et la richesse d'idée nécessaire à tout projet en gestation.

\section{Que peut apporter le HAUM à une cité de l'innovation collaborative ?}

Le point essentiel de toute la Cité de l'Innovation, la clé de voûte du projet complet, est incarné par les gens qui y vivront et qui y travailleront.
La Cité de l'Innovation, plus qu'un simple espace de travail doit être un lieu de vie où chacun se sent chez lui, dans un environnement propice à l'apparition de nouvelles idées.

Le HAUM, fournit le cadre pour que les idées nouvelles naissent et grandissent. L'association a dans son ADN la culture DIY nécessaire à l'innovation. (BullshitBingo, ndlr)

Toute la puissance d'un espace de rencontre, comme celui que la Cité de l'Innovation entend incarner, réside dans la communauté qui l'habite, l'anime et le transforme. Un fablab, au delà des seuls intérêts matériels qu'il représente (accès à des machines, des outils, etc...), est le lieu communautaire qui, modifiable par essence, ressemble à la communauté.

Le savoir faire apporté par le HAUM et les autres acteurs potentiels du fablab est pluridisciplinaire et basé sur l'expérimentation. Il renferme tous les ingrédients nécessaires pour passer d'une idée dans la tête d'un porteur de projet à un prototype sur le bureau du financeur.

Un projet incubé dans un fablab sera soumis continuellement à la critique : c'est un excellent moyen d'avancer rapidement. Le lab doit aussi être un endroit qui fourmille d'idées s'il veut développer, vivre et faire vivre l'innovation au sein de la Cité. Ainsi, le porteur de projet doit accepter d'échanger et de partager son savoir-faire et ses idées pour que le lab soit en mesure de répondre à ses attentes, par retours d'expériences. Cela n'est évidemment pas incompatible avec l'exploitation commerciale pour le porteur de projet. 

Dans le cadre des actions du fablab il y a bien évidement l'organisation d’événements à destination du public : conférences, ateliers, démonstrations, etc... L'intérêt d'une association comme le HAUM dans le projet de Cité de l'Innovation (et en particulier sur l'axe fablab), c'est la communauté qui gravite autour et les horaires étendus (soir et week-end) qui font vivre le lieu au delà des temps de travail.

\section{Que manque-t-il au HAUM pour offrir le meilleur à une cité de l'innovation ?}

Des outils de base pour un atelier polyvalent:

    machines électroportatives pour le travail du bois telles que scie circulaire, scie sauteuse, perceuse, visseuse 

    machines électroportatives pour le travail des métaux meuleuse d'angle Tronçonneuse a disque, scie a ruban ou alternative, poste a souder le metal

    outils de base clefs, tournevis, marteaux, scie a main

 
Des machines :

    Imprimante 3D (+ consommables : filament ou résine selon procédé, matière selon besoins de conception et de design)

    Fraiseuse numérique

    Tour numérique

    Découpeuse laser

    Découpeuse plasma

    Découpeuse vinyle


    Graveuse PCB

    Une hotte à flux laminaire (pour la sécurité et les opérations chimiques)


    Tricoteuse numérique

    Brodeuse numérique


    Tour de poterie

    Fonderie pour métaux moux (aluminium, laiton, cuivre)

    Machine pour transferts textiles

    Table de sérigraphie


    Imprimante (pour communication autour du HAUM par le HAUM, notamment)

    Plastifieuse

    Massicot automatique

    Perceuse à colonne

    Découpeuse à eau 

    Traceur pour imprimer des affiches grand format


Des outils :

    Trousse mécanique

    Outillages électronique et électrique (pinces coupantes, à dénuder, à sertir des cosses, etc)

    Oscilloscope

    Générateur de signaux

    Multimètres

    Programmeurs de cartes électroniques

    EPI

    Analyseur de spectre

    panoplie iFixit

    tableaux blanc / écran de vidéo projecteurs

    
Donc de la place pour ces machines et des Hommes pour s'en servir

Des lieux de rangements (verrouillés) pour projets/affaires... casiers

Matières premières :

    filament d'imprimante 3D (si imprimante à fil)

    plaques de bois pour découpe laser 

    plaques plastique pour usinage

    Métaux (aluminium, acier, cuivre,...)

    composants électroniques (composants de base, plaques de cuivre pour PCB, microcontrôleurs...) 

    composants pour montage électrique ou électronique (fils, câbles, cosses...)

    visserie-boulonnerie

    papier


Un lieu de stockage pour ces éléments ainsi que du mobilier (placards, …)

Différents espaces afin de permettre une meilleure collaboration : 
    - salle de conférence ouvrable au public (capacité 50-100 personnes minimum)
    - paillasses / établis pour bricoler (proximité avec évier)
    - salle de réunion : table ronde, prises pour ordinateur
    - espace cafet' (bar, tables), évier (plonge), micro-onde, frigo
    - espace détente, sofas, lumière tamisée

Lieux à équipés de connexion réseau (WiFi solide, câblage RJ45) (lumière, électricité, eau...)

Afin d'offrir un accompagnement efficace aux créateurs, le HAUM a besoin d'étendre ses plages d'ouverture. Aujourd'hui, les activités de chacun nous permettent d'ouvrir seulement le mardi et le jeudi soir de 18h30 à 1h (parfois plus). Ponctuellement, nous ouvrons le lieu le samedi, dimanche ou mercredi après-midi.
Idéalement, un lieu ouvert de manière quasi-permanente.

\section{Comment anime-t-on un fablab ?}
L'animation d'un fablab est un point crucial, en particulier sur les premiers mois/années. Le fablab, on l'a vu, s'articule autour d'une communauté qui s'améliore au cours du temps, il faut donc un élément premier, fédérateur de la communauté naissante.

Idéalement il y a deux axes majeurs dans l'animation d'un fablab : l'axe technique et l'axe communication.

La technique, tout d'abord, est l'objet, la raison d'être du lab. Le fablab étant un lieu où chacun vient pour réaliser des objets, il est bien sûr nécessaire que l'équipe animatrice soit compétente ; car oui, il faut une équipe.
La gestion d'un fablab comporte des points amusants, passionnants (projets, rencontres, formations, etc...) mais aussi des éléments plus complexes : gestion du stock de matières premières, maintenance/entretien des machines, relations extérieures, gestion des pannes/de l'utilisation des machines, etc...
Si elle ne requiert pas forcément une formation spécialisée, cette deuxième facette du fab manager (personne gérant le fablab) demande toutefois un investissement qu'on ne saurait attendre d'un bénévole.

Les animations à mettre en place du côté technique sont nombreuses : ateliers et formations sur les machines, mini-projets en groupe pour se former à certaines technologies, conférences sur divers sujets (électronique, matériaux, techniques de prototypage, etc...) mais aussi défis, challenges et autres activités permettant de transformer un ensemble d'usagers en une communauté soudée.

Il ne faut jamais perdre de vue que, au delà de sa mission technique, le fab manager cristallise l'âme du lieu : il re-motive ceux dont le projet patine, il aide de manière désintéressé quiconque le demande, il gère les éventuels conflits de personne et sert de médiateur en cas de besoin. L'aspect social du fab manager est essentiel et parmis tous les critères possibles, c'est celui qu'il faut mettre en avant dans un recrutement : on peut apprendre à se servir de machines mais on apprend que difficilement à composer avec l'aspect social d'une communauté partageant un lieu à longueur de journée.

Le deuxième axe de l'animation d'un fablab concerne la communication : le lab prend de la valeur au travers de sa communauté et des compétences qui se développent.

Une bonne communication, c'est l'assurance d'une entrée continue de nouvelles personnes, c'est l'assurance d'entrées dont on connait a priori une partie des envies. Une bonne communication c'est aussi une grande visibilité pour les occupants : article(s) sur leur projet, médiatisation, mise en avant et showcase, autant de choses essentielles pour la reconnaissance dont chaque porteur de projet a besoin.

Pour résumer, une animation de fablab requiert une ou deux personnes (la technique étant un poste à temps complet) qui se répartissent des tâches de communication et d'aide aux occupants. Le fablab peut (et doit) s'appuyer pour grandir sur un ensemble cohérent d’événements plus ou moins médiatiques axés sur les formations/conférences proposées mais aussi et surtout sur les projets en cours au lab. Dans les activités de communication, il est essentiel de participer à des événements (en Sarthe ou ailleurs) organisés par d'autres structures : dans la culture fablab, la reconnaissance vient des autres communautés.

\section{Quelles sont les interactions existantes avec d'autres structures ?}

Depuis sa création, le HAUM a su se faire connaître et reconnaître pour ses réalisations de qualité. Ce faisant, des acteurs de tous bords sont venus solliciter l'association pour qu'elle présente ses travaux, notamment lors de manifestations culturelles publiques. Grâce à ces événements, de nouveaux contacts se sont noués au fil du temps, ouvrant sur de nouveaux projets collaboratifs.

Ainsi, le HAUM œuvre avec des institutions publiques pour l'organisation d'événements variés, par exemple avec la Ruche numérique et l'ENSIM. Que ce soit l'organisation de conférences éclair hébergées par la Ruche ou la proposition de sujets pour les 24H du Code, il s'agit de s'impliquer dans l'événementiel local. De plus, l'association a participé à l'a mise en place des Repair Cafés.

Sur le plan associatif, le dHAUM est un exemple parlant d'interactions à l'initiative du HAUM. Fort de sa présence aux siestes Teriaki de 2014 avec Piano Stairs, le Hackerspace a démontré une nouvelle fois son inventivité pendant le festival Teriaki 2015 en proposant le dHAUM, cet ovni musical où l'on finit par oublier qui de l'utilisateur ou de la structure est le musicien. Ce projet a ainsi permis de renforcer des liens avec la compagnie Organic Orchestra.

D'autres événements sont co-organisés par le HAUM : les jeudis du libre avec Linux Maine, la journée de conférences AgileMans, etc.

Enfin, l'association cultive son réseau, établissant ou pérennisant ses contacts dans l'univers des fablabs et des hackerspaces (PiNG, /tmp/lab, Electrolab, etc). D'autres acteurs sont aussi en lien avec le HAUM : les petits débrouillards  par exemple, œuvrant pour la vulgarisation scientifique.  

Au cours de son développement, le HAUM s'est associé ponctuellement à d'autres structures et continuera à le faire.
Le premier type d'association concernait l'essaimage : participer à Teriaki ou à Festival D permet de faire connaître l'association au delà de son public cible et confronte les projets à l'appréciation de personnes extérieures. La co-organisation de RepairCafés, les discussions avec Organic Orchestra, etc... étaient autant de moyens de faire connaître le HAUM.
Le second type était axé sur l'organisation d’événements

\end{document}
